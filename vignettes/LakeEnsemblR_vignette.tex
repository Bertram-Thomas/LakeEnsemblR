% Options for packages loaded elsewhere
\PassOptionsToPackage{unicode}{hyperref}
\PassOptionsToPackage{hyphens}{url}
%
\documentclass[
]{article}
\usepackage{lmodern}
\usepackage{amssymb,amsmath}
\usepackage{ifxetex,ifluatex}
\ifnum 0\ifxetex 1\fi\ifluatex 1\fi=0 % if pdftex
  \usepackage[T1]{fontenc}
  \usepackage[utf8]{inputenc}
  \usepackage{textcomp} % provide euro and other symbols
\else % if luatex or xetex
  \usepackage{unicode-math}
  \defaultfontfeatures{Scale=MatchLowercase}
  \defaultfontfeatures[\rmfamily]{Ligatures=TeX,Scale=1}
\fi
% Use upquote if available, for straight quotes in verbatim environments
\IfFileExists{upquote.sty}{\usepackage{upquote}}{}
\IfFileExists{microtype.sty}{% use microtype if available
  \usepackage[]{microtype}
  \UseMicrotypeSet[protrusion]{basicmath} % disable protrusion for tt fonts
}{}
\makeatletter
\@ifundefined{KOMAClassName}{% if non-KOMA class
  \IfFileExists{parskip.sty}{%
    \usepackage{parskip}
  }{% else
    \setlength{\parindent}{0pt}
    \setlength{\parskip}{6pt plus 2pt minus 1pt}}
}{% if KOMA class
  \KOMAoptions{parskip=half}}
\makeatother
\usepackage{xcolor}
\IfFileExists{xurl.sty}{\usepackage{xurl}}{} % add URL line breaks if available
\IfFileExists{bookmark.sty}{\usepackage{bookmark}}{\usepackage{hyperref}}
\hypersetup{
  pdftitle={R package LakeEnsemblR: Basic Use and Sample Applications},
  pdfauthor={johannes.feldbauer@tu-dresden.de; tadhg.moore@dkit.ie; jorrit.mesman@unige.ch; ladwigjena@gmail.com},
  hidelinks,
  pdfcreator={LaTeX via pandoc}}
\urlstyle{same} % disable monospaced font for URLs
\usepackage[margin=1in]{geometry}
\usepackage{color}
\usepackage{fancyvrb}
\newcommand{\VerbBar}{|}
\newcommand{\VERB}{\Verb[commandchars=\\\{\}]}
\DefineVerbatimEnvironment{Highlighting}{Verbatim}{commandchars=\\\{\}}
% Add ',fontsize=\small' for more characters per line
\usepackage{framed}
\definecolor{shadecolor}{RGB}{248,248,248}
\newenvironment{Shaded}{\begin{snugshade}}{\end{snugshade}}
\newcommand{\AlertTok}[1]{\textcolor[rgb]{0.94,0.16,0.16}{#1}}
\newcommand{\AnnotationTok}[1]{\textcolor[rgb]{0.56,0.35,0.01}{\textbf{\textit{#1}}}}
\newcommand{\AttributeTok}[1]{\textcolor[rgb]{0.77,0.63,0.00}{#1}}
\newcommand{\BaseNTok}[1]{\textcolor[rgb]{0.00,0.00,0.81}{#1}}
\newcommand{\BuiltInTok}[1]{#1}
\newcommand{\CharTok}[1]{\textcolor[rgb]{0.31,0.60,0.02}{#1}}
\newcommand{\CommentTok}[1]{\textcolor[rgb]{0.56,0.35,0.01}{\textit{#1}}}
\newcommand{\CommentVarTok}[1]{\textcolor[rgb]{0.56,0.35,0.01}{\textbf{\textit{#1}}}}
\newcommand{\ConstantTok}[1]{\textcolor[rgb]{0.00,0.00,0.00}{#1}}
\newcommand{\ControlFlowTok}[1]{\textcolor[rgb]{0.13,0.29,0.53}{\textbf{#1}}}
\newcommand{\DataTypeTok}[1]{\textcolor[rgb]{0.13,0.29,0.53}{#1}}
\newcommand{\DecValTok}[1]{\textcolor[rgb]{0.00,0.00,0.81}{#1}}
\newcommand{\DocumentationTok}[1]{\textcolor[rgb]{0.56,0.35,0.01}{\textbf{\textit{#1}}}}
\newcommand{\ErrorTok}[1]{\textcolor[rgb]{0.64,0.00,0.00}{\textbf{#1}}}
\newcommand{\ExtensionTok}[1]{#1}
\newcommand{\FloatTok}[1]{\textcolor[rgb]{0.00,0.00,0.81}{#1}}
\newcommand{\FunctionTok}[1]{\textcolor[rgb]{0.00,0.00,0.00}{#1}}
\newcommand{\ImportTok}[1]{#1}
\newcommand{\InformationTok}[1]{\textcolor[rgb]{0.56,0.35,0.01}{\textbf{\textit{#1}}}}
\newcommand{\KeywordTok}[1]{\textcolor[rgb]{0.13,0.29,0.53}{\textbf{#1}}}
\newcommand{\NormalTok}[1]{#1}
\newcommand{\OperatorTok}[1]{\textcolor[rgb]{0.81,0.36,0.00}{\textbf{#1}}}
\newcommand{\OtherTok}[1]{\textcolor[rgb]{0.56,0.35,0.01}{#1}}
\newcommand{\PreprocessorTok}[1]{\textcolor[rgb]{0.56,0.35,0.01}{\textit{#1}}}
\newcommand{\RegionMarkerTok}[1]{#1}
\newcommand{\SpecialCharTok}[1]{\textcolor[rgb]{0.00,0.00,0.00}{#1}}
\newcommand{\SpecialStringTok}[1]{\textcolor[rgb]{0.31,0.60,0.02}{#1}}
\newcommand{\StringTok}[1]{\textcolor[rgb]{0.31,0.60,0.02}{#1}}
\newcommand{\VariableTok}[1]{\textcolor[rgb]{0.00,0.00,0.00}{#1}}
\newcommand{\VerbatimStringTok}[1]{\textcolor[rgb]{0.31,0.60,0.02}{#1}}
\newcommand{\WarningTok}[1]{\textcolor[rgb]{0.56,0.35,0.01}{\textbf{\textit{#1}}}}
\usepackage{longtable,booktabs}
% Correct order of tables after \paragraph or \subparagraph
\usepackage{etoolbox}
\makeatletter
\patchcmd\longtable{\par}{\if@noskipsec\mbox{}\fi\par}{}{}
\makeatother
% Allow footnotes in longtable head/foot
\IfFileExists{footnotehyper.sty}{\usepackage{footnotehyper}}{\usepackage{footnote}}
\makesavenoteenv{longtable}
\usepackage{graphicx,grffile}
\makeatletter
\def\maxwidth{\ifdim\Gin@nat@width>\linewidth\linewidth\else\Gin@nat@width\fi}
\def\maxheight{\ifdim\Gin@nat@height>\textheight\textheight\else\Gin@nat@height\fi}
\makeatother
% Scale images if necessary, so that they will not overflow the page
% margins by default, and it is still possible to overwrite the defaults
% using explicit options in \includegraphics[width, height, ...]{}
\setkeys{Gin}{width=\maxwidth,height=\maxheight,keepaspectratio}
% Set default figure placement to htbp
\makeatletter
\def\fps@figure{htbp}
\makeatother
\setlength{\emergencystretch}{3em} % prevent overfull lines
\providecommand{\tightlist}{%
  \setlength{\itemsep}{0pt}\setlength{\parskip}{0pt}}
\setcounter{secnumdepth}{5}
\usepackage{longtable}
% Make figures non-floating
\usepackage{float}
\usepackage{pdflscape}
\usepackage{amsmath}
\let\origfigure\figure
\let\endorigfigure\endfigure
\renewenvironment{figure}[1][2] {
    \expandafter\origfigure\expandafter[H]
} {
    \endorigfigure
}

\title{R package LakeEnsemblR: Basic Use and Sample Applications}
\author{\href{mailto:johannes.feldbauer@tu-dresden.de}{\nolinkurl{johannes.feldbauer@tu-dresden.de}} \and \href{mailto:tadhg.moore@dkit.ie}{\nolinkurl{tadhg.moore@dkit.ie}} \and \href{mailto:jorrit.mesman@unige.ch}{\nolinkurl{jorrit.mesman@unige.ch}} \and \href{mailto:ladwigjena@gmail.com}{\nolinkurl{ladwigjena@gmail.com}}}
\date{2020-04-25}

\begin{document}
\maketitle

{
\setcounter{tocdepth}{3}
\tableofcontents
}
\hypertarget{included-models}{%
\section{Included models}\label{included-models}}

LakeEnsemblR currently includes the following models: GLM (Hipsey et al.
(2019)), FLake (Mironov (2008)), GOTM (Umlauf, Bolding, and Burchard
(2005)), Simstrat (Goudsmit et al. (2002)), and MyLake (Saloranta and
Andersen (2007)).

\hypertarget{introduction}{%
\section{Introduction}\label{introduction}}

LakeEnsemblR is an R package that lets you run multiple one-dimensional
physical lake models.

The settings for a model run are controlled by one centralised,
``master'' configuration file in YAML format. In this configuration
file, you can set all the specifications for your model run, including
start and end time, time steps, links to meteorological forcing and
bathymetry files, etc. The package then converts these settings to the
configuration files required by each model, through the
\texttt{export\_config} function. This sets up all models to run with
the settings specified by the user, and the models are then run through
the \texttt{run\_ensemble} function. A netcdf file is created with the
outputs of all the models, and the package provides functions to extract
and plot this data.

Part of the design philosophy of LakeEnsemblR is that all input is given
in a standardized format. This entails standard column names (which
includes units), comma-separated ASCII files, and a DateTime format
using the international standard format (ISO 8601), which is
\texttt{YYYY-mm-dd\ HH:MM:SS}, for example
\texttt{2020-04-03\ 09:00:00}. In this document, we will explain what
the required files are and in what format they need to be. We also
advise you to look at the provided example files, and at the templates
provided with the R package (to be found in
\texttt{package/inst/extdata}, or extracted by the function
get\_template).

\hypertarget{installation}{%
\section{Installation}\label{installation}}

The code of LakeEnsemblR is hosted on the AEMON-J Github page
(\url{https://github.com/aemon-j/LakeEnsemblR}), and can be installed
using the \texttt{devtools} package

\begin{verbatim}
devtools::install_github("aemon-j/LakeEnsemblR)
\end{verbatim}

The package relies on multiple other packages that also need to be
installed. Most notably, to run the multiple models, it requires the
packages FLakeR, GLM3r, GOTMr, SimstratR, and MyLakeR. These packages
run the individual models, and contain ways of running the models for
the platforms Windows, MacOS, or UNIX, through either executables, or by
having the model code in R.

\hypertarget{the-lakeensemblr-configuration-file}{%
\section{The LakeEnsemblR configuration
file}\label{the-lakeensemblr-configuration-file}}

In this section, we go through the LakeEnsemblR configuration file. This
file controls the settings with which the models are run. It is written
in YAML (Yaml Ain't Markup Language) format, and can be opened by
text-editors such as Notepad or Notepad++. Although not needed to use
LakeEnsemblR, you can read this file into R with the \texttt{configr}
package (\texttt{read.config} function). Within LakeEnsemblR, we provide
the \texttt{get\_yaml\_value} and \texttt{input\_yaml} functions to get
and input values into this file type.

There is an example LakeEnsemblR configuration file provided in the
example dataset in the package
(\texttt{LakeEnsemblR::get\_template("LakeEnsemblR\_config")}) or you
can download a copy from GitHub
\href{https://github.com/aemon-j/LakeEnsemblR/blob/master/inst/extdata/feeagh/Feeagh_master_config.yaml}{here}.

\hypertarget{location}{%
\subsection{Location}\label{location}}

The first section is ``Location''. Here you specify the name, latitude
and longitude, elevation, maximum depth, and initial depth.

You also need to provide a link to the hypsograph (i.e.~surface area per
depth) file. As in the rest of the configuration file, all links to
files are relative to the \texttt{folder} argument in the LakeEnsemblR
functions (default is the R working directory). We strongly advise to
set the working directory to the location of the LER config file, and
link to the files relative to this directory (we explain this further in
the chapter ``Workflow'' of this vignette). For example, if your
hypsograph file is called \texttt{hypsograph.csv} and in the same folder
as the LER config file, the corresponding line in the LER config file
would look like

\begin{verbatim}
  hypsograph: hypsograph.csv
\end{verbatim}

The data needs to be a comma-separated file (.csv) where 0m is the
surface and all depths are reported as positive, in meters. Area needs
to be in meters squared. The column names \emph{must} be
\texttt{Depth\_meter} and \texttt{Area\_meterSquared}

Example of data:

\begin{verbatim}
Depth_meter,Area_meterSquared
0,3931000
1,3688025
2,3445050
3,3336093.492
4,3225992.455
5,3133491.11
6,3029720
...
\end{verbatim}

\hypertarget{time}{%
\subsection{Time}\label{time}}

In the ``Time'' section, you fill in the start and end date of the
simulation, and the model time step. \texttt{time\_step} indicates the
model integration time step, so each time step that the model performs a
calculation.

\hypertarget{config-files}{%
\subsection{Config files}\label{config-files}}

In this section, you link to the model-specific configuration files.
Templates of these can be found in in the package
(\texttt{get\_template("FLake\_config")},
\texttt{get\_template("GLM\_config")}, etc.) or you can download a copy
from GitHub
\href{https://github.com/aemon-j/LakeEnsemblR/blob/master/inst/extdata}{here}.
The setup of LakeEnsemblR is such that you will usually not have to
access these files, as most settings are regulated through the LER
``master'' config file. However, should you want to change some of the
more specialised settings in each model, that is possible.

\hypertarget{light}{%
\subsection{Light}\label{light}}

Here you can either give a value for Kw (light extinction coefficient)
in 1/m, or give the link to a file, where you can vary Kw over time
(template available).

\hypertarget{observations}{%
\subsection{Observations}\label{observations}}

If you have observations of water temperature or ice cover, you can fill
them in here. These will be used in plotting, and in case of water
temperature potentially in initialising the temperature profile at the
start of the simulation (but see next section).

For water temperature, the data needs to be a comma separated values
(.csv) file where the datetime column is in the format
\texttt{YYYY-mm-dd\ HH:MM:SS}. Depths are positive and relative to the
water surface. Water temperature is in degrees Celsius. The column names
\emph{must} be \texttt{datetime}, \texttt{Depth\_meter} and
\texttt{Water\_Temperature\_celsius} (templates available).

Example of data:

\begin{verbatim}
datetime,Depth_meter,Water_Temperature_celsius
2004-01-05 00:00:00,0.9,6.97
2004-01-05 00:00:00,2.5,6.71
2004-01-05 00:00:00,5,6.73
2004-01-05 00:00:00,8,6.76
...
\end{verbatim}

\hypertarget{input}{%
\subsection{Input}\label{input}}

In the ``Input'' section, you give information about the initial
temperature profile, meteorological forcing, and ice.

Firstly, you can give the initial temperature profile with which to
start the simulation (link to .csv file, template available). If you
leave it empty, the water temperature observations will be used,
provided you have an observation on the starting time of your
simulation.

Secondly, you give the link to the file with your meteorological data.
The data needs to be a comma separated values (.csv) file where the
datetime column is in the format \texttt{YYYY-mm-dd\ HH:MM:SS}. See
table 1 for the list of variables, units and column names. The
time\_zone setting has not been implemented yet.

Table 1. Description of meteorological variables used within
LakeEnsemblR with units and required column names.

\begin{longtable}[]{@{}llll@{}}
\toprule
\begin{minipage}[b]{0.12\columnwidth}\raggedright
Description\strut
\end{minipage} & \begin{minipage}[b]{0.03\columnwidth}\raggedright
Units\strut
\end{minipage} & \begin{minipage}[b]{0.18\columnwidth}\raggedright
Column Name\strut
\end{minipage} & \begin{minipage}[b]{0.56\columnwidth}\raggedright
Status\strut
\end{minipage}\tabularnewline
\midrule
\endhead
\begin{minipage}[t]{0.12\columnwidth}\raggedright
Downwelling longwave radaiation\strut
\end{minipage} & \begin{minipage}[t]{0.03\columnwidth}\raggedright
W/m2\strut
\end{minipage} & \begin{minipage}[t]{0.18\columnwidth}\raggedright
Longwave\_Radiation\_Downwelling\_wattPerMeterSquared\strut
\end{minipage} & \begin{minipage}[t]{0.56\columnwidth}\raggedright
If not provided,it is calculated internally from air temperature, cloud
cover and relative humidity/dewpoint temperature\strut
\end{minipage}\tabularnewline
\begin{minipage}[t]{0.12\columnwidth}\raggedright
Downwelling shortwave radaiation\strut
\end{minipage} & \begin{minipage}[t]{0.03\columnwidth}\raggedright
W/m2\strut
\end{minipage} & \begin{minipage}[t]{0.18\columnwidth}\raggedright
Shortwave\_Radiation\_Downwelling\_wattPerMeterSquared\strut
\end{minipage} & \begin{minipage}[t]{0.56\columnwidth}\raggedright
Required\strut
\end{minipage}\tabularnewline
\begin{minipage}[t]{0.12\columnwidth}\raggedright
Cloud cover\strut
\end{minipage} & \begin{minipage}[t]{0.03\columnwidth}\raggedright
-\strut
\end{minipage} & \begin{minipage}[t]{0.18\columnwidth}\raggedright
Cloud\_Cover\_decimalFraction\strut
\end{minipage} & \begin{minipage}[t]{0.56\columnwidth}\raggedright
If not provided,it is calculated internally from air temperature,
short-wave radiation, latitude, longitude, elevation and relative
humidity/dewpoint temperature\strut
\end{minipage}\tabularnewline
\begin{minipage}[t]{0.12\columnwidth}\raggedright
Air temperature\strut
\end{minipage} & \begin{minipage}[t]{0.03\columnwidth}\raggedright
°C\strut
\end{minipage} & \begin{minipage}[t]{0.18\columnwidth}\raggedright
Air\_Temperature\_celsius\strut
\end{minipage} & \begin{minipage}[t]{0.56\columnwidth}\raggedright
Required\strut
\end{minipage}\tabularnewline
\begin{minipage}[t]{0.12\columnwidth}\raggedright
Relative humidity\strut
\end{minipage} & \begin{minipage}[t]{0.03\columnwidth}\raggedright
\%\strut
\end{minipage} & \begin{minipage}[t]{0.18\columnwidth}\raggedright
Relative\_Humidity\_percent\strut
\end{minipage} & \begin{minipage}[t]{0.56\columnwidth}\raggedright
If not provided,it is calculated internally from air temperature and
dewpoint temperature\strut
\end{minipage}\tabularnewline
\begin{minipage}[t]{0.12\columnwidth}\raggedright
Dewpoint temperature\strut
\end{minipage} & \begin{minipage}[t]{0.03\columnwidth}\raggedright
°C\strut
\end{minipage} & \begin{minipage}[t]{0.18\columnwidth}\raggedright
Dewpoint\_Temperature\_celsius\strut
\end{minipage} & \begin{minipage}[t]{0.56\columnwidth}\raggedright
If not provided,it is calculated internally from air temperature and
relative humidity\strut
\end{minipage}\tabularnewline
\begin{minipage}[t]{0.12\columnwidth}\raggedright
Wind speed at 10m\strut
\end{minipage} & \begin{minipage}[t]{0.03\columnwidth}\raggedright
m/s\strut
\end{minipage} & \begin{minipage}[t]{0.18\columnwidth}\raggedright
Ten\_Meter\_Elevation\_Wind\_Speed\_meterPerSecond\strut
\end{minipage} & \begin{minipage}[t]{0.56\columnwidth}\raggedright
Either wind speed or u and v vectors is required\strut
\end{minipage}\tabularnewline
\begin{minipage}[t]{0.12\columnwidth}\raggedright
Wind direction at 10m\strut
\end{minipage} & \begin{minipage}[t]{0.03\columnwidth}\raggedright
°C\strut
\end{minipage} & \begin{minipage}[t]{0.18\columnwidth}\raggedright
Ten\_Meter\_Elevation\_Wind\_Direction\_degree\strut
\end{minipage} & \begin{minipage}[t]{0.56\columnwidth}\raggedright
Not required, but if provided u and v vectors are calculated
internally\strut
\end{minipage}\tabularnewline
\begin{minipage}[t]{0.12\columnwidth}\raggedright
Wind u-vector at 10m\strut
\end{minipage} & \begin{minipage}[t]{0.03\columnwidth}\raggedright
m/s\strut
\end{minipage} & \begin{minipage}[t]{0.18\columnwidth}\raggedright
Ten\_Meter\_Uwind\_vector\_meterPerSecond\strut
\end{minipage} & \begin{minipage}[t]{0.56\columnwidth}\raggedright
Either wind speed or u and v vectors is required\strut
\end{minipage}\tabularnewline
\begin{minipage}[t]{0.12\columnwidth}\raggedright
Wind v-vector at 10m\strut
\end{minipage} & \begin{minipage}[t]{0.03\columnwidth}\raggedright
m/s\strut
\end{minipage} & \begin{minipage}[t]{0.18\columnwidth}\raggedright
Ten\_Meter\_Vwind\_vector\_meterPerSecond\strut
\end{minipage} & \begin{minipage}[t]{0.56\columnwidth}\raggedright
Either wind speed or u and v vectors is required\strut
\end{minipage}\tabularnewline
\begin{minipage}[t]{0.12\columnwidth}\raggedright
Precipitation\strut
\end{minipage} & \begin{minipage}[t]{0.03\columnwidth}\raggedright
m/s\strut
\end{minipage} & \begin{minipage}[t]{0.18\columnwidth}\raggedright
Precipitation\_meterPerSecond\strut
\end{minipage} & \begin{minipage}[t]{0.56\columnwidth}\raggedright
Not strictly required but is important for mass budgets in some
models\strut
\end{minipage}\tabularnewline
\begin{minipage}[t]{0.12\columnwidth}\raggedright
Rainfall\strut
\end{minipage} & \begin{minipage}[t]{0.03\columnwidth}\raggedright
m/s\strut
\end{minipage} & \begin{minipage}[t]{0.18\columnwidth}\raggedright
Rainfall\_meterPerSecond\strut
\end{minipage} & \begin{minipage}[t]{0.56\columnwidth}\raggedright
Required\strut
\end{minipage}\tabularnewline
\begin{minipage}[t]{0.12\columnwidth}\raggedright
Snowfall\strut
\end{minipage} & \begin{minipage}[t]{0.03\columnwidth}\raggedright
m/day\strut
\end{minipage} & \begin{minipage}[t]{0.18\columnwidth}\raggedright
Snowfall\_meterPerDay\strut
\end{minipage} & \begin{minipage}[t]{0.56\columnwidth}\raggedright
If not provided,it is calculated internally from rain when air
temperature \textless{} 0 degC\strut
\end{minipage}\tabularnewline
\begin{minipage}[t]{0.12\columnwidth}\raggedright
Sea level pressure\strut
\end{minipage} & \begin{minipage}[t]{0.03\columnwidth}\raggedright
Pa\strut
\end{minipage} & \begin{minipage}[t]{0.18\columnwidth}\raggedright
Sea\_Level\_Barometric\_Pressure\_pascal\strut
\end{minipage} & \begin{minipage}[t]{0.56\columnwidth}\raggedright
Not required\strut
\end{minipage}\tabularnewline
\begin{minipage}[t]{0.12\columnwidth}\raggedright
Surface level pressure\strut
\end{minipage} & \begin{minipage}[t]{0.03\columnwidth}\raggedright
Pa\strut
\end{minipage} & \begin{minipage}[t]{0.18\columnwidth}\raggedright
Surface\_Level\_Barometric\_Pressure\_pascal\strut
\end{minipage} & \begin{minipage}[t]{0.56\columnwidth}\raggedright
Required\strut
\end{minipage}\tabularnewline
\begin{minipage}[t]{0.12\columnwidth}\raggedright
Vapour pressure\strut
\end{minipage} & \begin{minipage}[t]{0.03\columnwidth}\raggedright
mbar\strut
\end{minipage} & \begin{minipage}[t]{0.18\columnwidth}\raggedright
Vapor\_Pressure\_milliBar\strut
\end{minipage} & \begin{minipage}[t]{0.56\columnwidth}\raggedright
If not provided,it is calculated internally from air temperature and
relative humidity/dewpoint temperature\strut
\end{minipage}\tabularnewline
\bottomrule
\end{longtable}

Lastly, you can specify if you want to use the ice modules that are
present in some of the models.

\hypertarget{inflows}{%
\subsection{Inflows}\label{inflows}}

Specify if you want to add inflows to your simulation. If yes, you need
to link to a file with the inflow data. The data needs to be a comma
separated values (.csv) file where the datetime column is in the format
\texttt{YYYY-mm-dd\ HH:MM:SS}. Flow discharge, water temperature, and
salinity need to be specified. The column names \emph{must} be
\texttt{datetime}, \texttt{Flow\_metersCubedPerSecond},
\texttt{Water\_Temperature\_celsius}, and
\texttt{Salinity\_practicalSalinityUnits} (template available).

Example of data:

\begin{verbatim}
datetime,Flow_metersCubedPerSecond,Water_Temperature_celsius,Salinity_practicalSalinityUnits
2005-01-01 00:00:00,5.62,6.96,0.00
2005-01-02 00:00:00,2.32,6.00,0.00
2005-01-03 00:00:00,1.77,8.44,0.00
2005-01-04 00:00:00,4.64,7.27,0.00
...
\end{verbatim}

\hypertarget{output}{%
\subsection{Output}\label{output}}

In the ``Output'' section, you specify how the output should look like.
Currently, only the netcdf option is in place. You can specify the depth
interval of the output, and the frequency. Also specify what variables
should be in the output (currently only ``temp'' and ``ice\_height'').

\hypertarget{biogeochemistry}{%
\subsection{Biogeochemistry}\label{biogeochemistry}}

Currently not implemented

\hypertarget{model-parameters}{%
\subsection{Model parameters}\label{model-parameters}}

All models in LakeEnsemblR have different parameterizations. In this
section, you can set the value of any parameter in one of the
model-specific configuration files.

You can give either only the name of the parameter, but if needed you
can also provide the name of the section in which the parameter occurs,
separated by a \texttt{/}. For example for GOTM's k\_min parameter:

\begin{verbatim}
  GOTM:
    k_min: 3.6E-6
\end{verbatim}

or:

\begin{verbatim}
  GOTM:
    turb_param/k_min: 3.6E-6
\end{verbatim}

\hypertarget{calibration}{%
\subsection{Calibration}\label{calibration}}

This section is used when calibrating the models. In the package, the
run\_LHC (Latin-Hypercube) and run\_MCMC (Monte-Carlo Markov Chain)
functions do this, but this is currently still in a beta-phase and this
will be more clearly explained in a later version of this vignette.

\hypertarget{workflow}{%
\section{Workflow}\label{workflow}}

In this chapter, we quickly show you how your workflow with LakeEnsemblR
could look like.

\hypertarget{setting-up-a-directory}{%
\subsection{Setting up a directory}\label{setting-up-a-directory}}

First, you make an empty directory for the simulations of a specific
lake. In this directory, you place the LakeEnsemblR configuration file,
and the necessary LakeEnsemblR-format input files (e.g.~meteorology,
inflow, water temperature observations, hypsograph). Within this
directory, you create empty folders for each model (FLake, GLM, GOTM,
Simstrat, MyLake), and in here you place the model-specific
configuration files, corresponding the information you put in the
\texttt{config\_files} section in the LER config file.

\hypertarget{export_config}{%
\subsection{Export\_config}\label{export_config}}

Then, you run the \texttt{export\_config} function, which exports the
settings in the LER configuration file to the model-specific
configuration files. It is possible to run parts of this function, such
as \texttt{export\_meteo}, if the meteorological forcing is the only
thing you changed, but usually you will run \texttt{export\_config}.

\hypertarget{optional-changes-in-model-specific-directories}{%
\subsection{Optional: changes in model-specific
directories}\label{optional-changes-in-model-specific-directories}}

Optionally, you can now go into the model-specific configuration files
and make further changes. This should not be needed for regular
simulations, especially since you can control parameters in the
\texttt{model\_parameters} section of the LER config file, but the
option is there. For example, if you want to change the depth of the
inflow in the Simstrat model, you could go to the Simstrat folders and
make these changes manually (or through a script you wrote). It speaks
for itself that these changes should be done only if you are sure this
is what you need, and LakeEnsemblR does not provide further support for
this.

\hypertarget{run_ensemble}{%
\subsection{Run\_ensemble}\label{run_ensemble}}

The next step would be to call \texttt{run\_ensemble}, which runs all
the models. A new folder called ``output'' is created, in which a
netcdf-file will be put with all the results from the model runs. If you
are interested, each model sub-folder will also have an ``output''
folder, with the model-specific model output.

\hypertarget{post-processing}{%
\subsection{Post-processing}\label{post-processing}}

Now that the model simulations are finished, you can extract the data
with \texttt{load\_var}, or plot it with \texttt{plot\_ensemble}. You
can extract data from the netcdf file to do any further analysis. The
\texttt{ncdf4} R package supports working with netcdf data in R, and the
PyNcView software (\url{https://sourceforge.net/projects/pyncview/}) can
be used to look at netcdf output.

We are currently working on a function to add new simulations (for
example with a different meteorology file) to the same netcdf and
including them in the analysis.

\hypertarget{example-model-run}{%
\subsection{Example model run}\label{example-model-run}}

See below for an example run, and some post-processing of the data.

\begin{Shaded}
\begin{Highlighting}[]
\CommentTok{# Install packages - Ensure all packages are up to date - parallel development ongoing}
\CommentTok{#install.packages("devtools")}
\NormalTok{devtools}\OperatorTok{::}\KeywordTok{install_github}\NormalTok{(}\StringTok{"GLEON/GLM3r"}\NormalTok{)}
\NormalTok{devtools}\OperatorTok{::}\KeywordTok{install_github}\NormalTok{(‘USGS}\OperatorTok{-}\NormalTok{R}\OperatorTok{/}\NormalTok{glmtools’, }\DataTypeTok{ref =}\NormalTok{ ‘ggplot_overhaul’)}
\NormalTok{devtools}\OperatorTok{::}\KeywordTok{install_github}\NormalTok{(}\StringTok{"aemon-j/FLakeR"}\NormalTok{, }\DataTypeTok{ref =}\NormalTok{ “inflow”)}
\NormalTok{devtools}\OperatorTok{::}\KeywordTok{install_github}\NormalTok{(}\StringTok{"aemon-j/GOTMr"}\NormalTok{)}
\NormalTok{devtools}\OperatorTok{::}\KeywordTok{install_github}\NormalTok{(}\StringTok{"aemon-j/gotmtools"}\NormalTok{)}
\NormalTok{devtools}\OperatorTok{::}\KeywordTok{install_github}\NormalTok{(}\StringTok{"aemon-j/SimstratR"}\NormalTok{)}
\NormalTok{devtools}\OperatorTok{::}\KeywordTok{install_github}\NormalTok{(}\StringTok{"aemon-j/LakeEnsemblR"}\NormalTok{)}
\NormalTok{devtools}\OperatorTok{::}\KeywordTok{install_github}\NormalTok{(}\StringTok{"aemon-j/MyLakeR"}\NormalTok{)}

\CommentTok{# Load libraries}
\KeywordTok{library}\NormalTok{(gotmtools)}
\KeywordTok{library}\NormalTok{(LakeEnsemblR)}

\CommentTok{# Set working directory with example data from Lough Feeagh, Ireland}
\NormalTok{template_folder <-}\StringTok{ }\KeywordTok{system.file}\NormalTok{(}\StringTok{"extdata/feeagh"}\NormalTok{, }\DataTypeTok{package=} \StringTok{"LakeEnsemblR"}\NormalTok{)}
\KeywordTok{dir.create}\NormalTok{(}\StringTok{"example"}\NormalTok{) }\CommentTok{# Create example folder}
\KeywordTok{file.copy}\NormalTok{(}\DataTypeTok{from =}\NormalTok{ template_folder, }\DataTypeTok{to =} \StringTok{"example"}\NormalTok{, }\DataTypeTok{recursive =} \OtherTok{TRUE}\NormalTok{)}
\KeywordTok{setwd}\NormalTok{(}\StringTok{"example/feeagh"}\NormalTok{) }\CommentTok{# Change working directory to example folder}

\CommentTok{# Set models & config file}
\NormalTok{model <-}\StringTok{ }\KeywordTok{c}\NormalTok{(}\StringTok{"GLM"}\NormalTok{,  }\StringTok{"FLake"}\NormalTok{, }\StringTok{"GOTM"}\NormalTok{, }\StringTok{"Simstrat"}\NormalTok{, }\StringTok{"MyLake"}\NormalTok{)}
\NormalTok{config_file <-}\StringTok{ "Feeagh_master_config.yaml"}

\CommentTok{# 1. Example - creates directories with all model setup}
\KeywordTok{export_config}\NormalTok{(}\DataTypeTok{config_file =}\NormalTok{ config_file, }\DataTypeTok{model =}\NormalTok{ model, }\DataTypeTok{folder =} \StringTok{"."}\NormalTok{)}

\CommentTok{# 2. Create meteo driver files}
\KeywordTok{export_meteo}\NormalTok{(}\DataTypeTok{config_file =}\NormalTok{ config_file, }\DataTypeTok{model =}\NormalTok{ model)}

\CommentTok{# 3. Create initial conditions}
\NormalTok{start_date <-}\StringTok{ }\KeywordTok{get_yaml_value}\NormalTok{(}\DataTypeTok{file =}\NormalTok{ config_file, }\DataTypeTok{label =}  \StringTok{"time"}\NormalTok{, }\DataTypeTok{key =} \StringTok{"start"}\NormalTok{)}

\KeywordTok{export_init_cond}\NormalTok{(}\DataTypeTok{config_file =}\NormalTok{ config_file, }
                 \DataTypeTok{model =}\NormalTok{ model,}
                 \DataTypeTok{date =}\NormalTok{ start_date,}
                 \DataTypeTok{print =} \OtherTok{TRUE}\NormalTok{)}

\CommentTok{# 4. Run ensemble lake models}
\NormalTok{wtemp_list <-}\StringTok{ }\KeywordTok{run_ensemble}\NormalTok{(}\DataTypeTok{config_file =}\NormalTok{ config_file,}
                           \DataTypeTok{model =} \KeywordTok{c}\NormalTok{(}\StringTok{"FLake"}\NormalTok{, }\StringTok{"GLM"}\NormalTok{, }\StringTok{"GOTM"}\NormalTok{, }\StringTok{"Simstrat"}\NormalTok{, }\StringTok{"MyLake"}\NormalTok{),}
                           \DataTypeTok{return_list =} \OtherTok{TRUE}\NormalTok{)}
\end{Highlighting}
\end{Shaded}

Post-processing:

\begin{Shaded}
\begin{Highlighting}[]
\CommentTok{# Load libraries for post-processing}
\KeywordTok{library}\NormalTok{(ggpubr)}
\KeywordTok{library}\NormalTok{(ggplot2)}

\CommentTok{## Plot model output using gotmtools/ggplot2}

\CommentTok{# Extract names of all the variables in netCDF}
\NormalTok{ens_out <-}\StringTok{ "output/ensemble_output.nc"}
\NormalTok{vars <-}\StringTok{ }\NormalTok{gotmtools}\OperatorTok{::}\KeywordTok{list_vars}\NormalTok{(ens_out)}
\NormalTok{vars }\CommentTok{# Print variables}

\NormalTok{plist <-}\StringTok{ }\KeywordTok{list}\NormalTok{() }\CommentTok{# Initialize empty list for storing plots of each variable}
\ControlFlowTok{for}\NormalTok{(i }\ControlFlowTok{in} \DecValTok{1}\OperatorTok{:}\DecValTok{5}\NormalTok{)\{}
\NormalTok{  p1 <-}\StringTok{ }\NormalTok{gotmtools}\OperatorTok{::}\KeywordTok{plot_vari}\NormalTok{(}\DataTypeTok{ncdf =}\NormalTok{ ens_out,}
                             \DataTypeTok{var =}\NormalTok{ vars[i],}
                             \DataTypeTok{incl_time =} \OtherTok{FALSE}\NormalTok{,}
                             \DataTypeTok{limits =} \KeywordTok{c}\NormalTok{(}\DecValTok{0}\NormalTok{,}\DecValTok{25}\NormalTok{),}
                             \DataTypeTok{zlab =} \StringTok{"degC"}\NormalTok{)}
\NormalTok{  p1 <-}\StringTok{ }\NormalTok{p1 }\OperatorTok{+}\StringTok{ }\KeywordTok{scale_y_reverse}\NormalTok{() }\OperatorTok{+}\StringTok{ }\CommentTok{#Reverse y-axis}
\StringTok{    }\KeywordTok{coord_cartesian}\NormalTok{(}\DataTypeTok{ylim =} \KeywordTok{c}\NormalTok{(}\DecValTok{45}\NormalTok{,}\DecValTok{0}\NormalTok{))}\OperatorTok{+}\StringTok{ }\CommentTok{# ggplot2 v3.3 is sensitive to order of ylim}
\StringTok{    }\KeywordTok{ggtitle}\NormalTok{(vars[i]) }\OperatorTok{+}\StringTok{ }\CommentTok{# Add title using variable name}
\StringTok{    }\KeywordTok{xlab}\NormalTok{(}\StringTok{""}\NormalTok{)}\OperatorTok{+}\StringTok{ }\CommentTok{# Remove x-label}
\StringTok{    }\KeywordTok{theme_bw}\NormalTok{(}\DataTypeTok{base_size =} \DecValTok{18}\NormalTok{) }\CommentTok{# Increase font size of plots}
\NormalTok{  plist[[i]] <-}\StringTok{ }\NormalTok{p1}
\NormalTok{\}}

\CommentTok{# Plot all model simulations}
\CommentTok{# install.packages("ggpubr")}
\NormalTok{g1 <-}\StringTok{ }\NormalTok{ggpubr}\OperatorTok{::}\KeywordTok{ggarrange}\NormalTok{(}\DataTypeTok{plotlist =}\NormalTok{ plist, }\DataTypeTok{ncol =} \DecValTok{1}\NormalTok{, }\DataTypeTok{common.legend =} \OtherTok{TRUE}\NormalTok{, }\DataTypeTok{legend =} \StringTok{"right"}\NormalTok{)}
\NormalTok{g1}
\KeywordTok{ggsave}\NormalTok{(}\StringTok{"output/model_ensemble_watertemp.png"}\NormalTok{, g1,  }\DataTypeTok{dpi =} \DecValTok{300}\NormalTok{,}\DataTypeTok{width =} \DecValTok{384}\NormalTok{,}\DataTypeTok{height =} \DecValTok{300}\NormalTok{, }\DataTypeTok{units =} \StringTok{"mm"}\NormalTok{)}
\end{Highlighting}
\end{Shaded}

\hypertarget{citation}{%
\section{Citation}\label{citation}}

See

\begin{verbatim}
citation("LakeEnsemblR")
\end{verbatim}

on how to cite this project.

Although this information is included when running the function above,
we would like to repeat that in case you use and cite LakeEnsemblR, you
will also need to cite the individual models that you used in your
simulations.

\hypertarget{references}{%
\section*{References}\label{references}}
\addcontentsline{toc}{section}{References}

\hypertarget{refs}{}
\leavevmode\hypertarget{ref-goudsmit_application_2002}{}%
Goudsmit, G.-H., H. Burchard, F. Peeters, and A. Wüest. 2002.
``Application of \(k-\epsilon\) Turbulence Models to Enclosed Basins:
The Role of Internal Seiches.'' \emph{Journal of Geophysical Research:
Oceans} 107 (C12): 23--21--23--13.
\url{https://doi.org/10.1029/2001JC000954}.

\leavevmode\hypertarget{ref-hipsey_general_2019}{}%
Hipsey, M. R., L. C. Bruce, C. Boon, B. Busch, C. C. Carey, D. P.
Hamilton, P. C. Hanson, et al. 2019. ``A General Lake Model (GLM 3.0)
for Linking with High-Frequency Sensor Data from the Global Lake
Ecological Observatory Network (GLEON).'' \emph{Geoscientific Model
Development} 12 (1): 473--523.
\url{https://doi.org/10.5194/gmd-12-473-2019}.

\leavevmode\hypertarget{ref-mironov_flake_2008}{}%
Mironov, Dimitrii V. 2008. \emph{Parameterization of Lakes in Numerical
Weather Prediction. Description of a Lake Model}. COSMO Technical
Report. Offenbach am Main, Germany: Deutscher Wetterdienst.
\url{http://www.flake.igb-berlin.de/site/doc}.

\leavevmode\hypertarget{ref-saloranta_mylakemulti_year_2007}{}%
Saloranta, Tuomo M., and Tom Andersen. 2007. ``MyLake---A Multi-Year
Lake Simulation Model Code Suitable for Uncertainty and Sensitivity
Analysis Simulations.'' \emph{Ecological Modelling}, Uncertainty in
Ecological Models, 207 (1): 45--60.
\url{https://doi.org/10.1016/j.ecolmodel.2007.03.018}.

\leavevmode\hypertarget{ref-umlauf_gotm_2005}{}%
Umlauf, Lars, Karsten Bolding, and Hans Burchard. 2005. \emph{GOTM --
Scientific Documentation} (version 3.2). Marine Science Reports.
Warnemuende, Germany: Leibniz-Institute for Baltic Sea Research.
\url{https://gotm.net/portfolio/documentation/}.

\end{document}
